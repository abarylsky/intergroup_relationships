\documentclass[man]{apa6}
\usepackage{lmodern}
\usepackage{amssymb,amsmath}
\usepackage{ifxetex,ifluatex}
\usepackage{fixltx2e} % provides \textsubscript
\ifnum 0\ifxetex 1\fi\ifluatex 1\fi=0 % if pdftex
  \usepackage[T1]{fontenc}
  \usepackage[utf8]{inputenc}
\else % if luatex or xelatex
  \ifxetex
    \usepackage{mathspec}
  \else
    \usepackage{fontspec}
  \fi
  \defaultfontfeatures{Ligatures=TeX,Scale=MatchLowercase}
\fi
% use upquote if available, for straight quotes in verbatim environments
\IfFileExists{upquote.sty}{\usepackage{upquote}}{}
% use microtype if available
\IfFileExists{microtype.sty}{%
\usepackage{microtype}
\UseMicrotypeSet[protrusion]{basicmath} % disable protrusion for tt fonts
}{}
\usepackage{hyperref}
\hypersetup{unicode=true,
            pdftitle={The title},
            pdfauthor={First Author~\& Ernst-August Doelle},
            pdfkeywords={keywords},
            pdfborder={0 0 0},
            breaklinks=true}
\urlstyle{same}  % don't use monospace font for urls
\usepackage{graphicx,grffile}
\makeatletter
\def\maxwidth{\ifdim\Gin@nat@width>\linewidth\linewidth\else\Gin@nat@width\fi}
\def\maxheight{\ifdim\Gin@nat@height>\textheight\textheight\else\Gin@nat@height\fi}
\makeatother
% Scale images if necessary, so that they will not overflow the page
% margins by default, and it is still possible to overwrite the defaults
% using explicit options in \includegraphics[width, height, ...]{}
\setkeys{Gin}{width=\maxwidth,height=\maxheight,keepaspectratio}
\IfFileExists{parskip.sty}{%
\usepackage{parskip}
}{% else
\setlength{\parindent}{0pt}
\setlength{\parskip}{6pt plus 2pt minus 1pt}
}
\setlength{\emergencystretch}{3em}  % prevent overfull lines
\providecommand{\tightlist}{%
  \setlength{\itemsep}{0pt}\setlength{\parskip}{0pt}}
\setcounter{secnumdepth}{0}
% Redefines (sub)paragraphs to behave more like sections
\ifx\paragraph\undefined\else
\let\oldparagraph\paragraph
\renewcommand{\paragraph}[1]{\oldparagraph{#1}\mbox{}}
\fi
\ifx\subparagraph\undefined\else
\let\oldsubparagraph\subparagraph
\renewcommand{\subparagraph}[1]{\oldsubparagraph{#1}\mbox{}}
\fi

%%% Use protect on footnotes to avoid problems with footnotes in titles
\let\rmarkdownfootnote\footnote%
\def\footnote{\protect\rmarkdownfootnote}


  \title{The title}
    \author{First Author\textsuperscript{1}~\& Ernst-August Doelle\textsuperscript{1,2}}
    \date{}
  
\shorttitle{Title}
\affiliation{
\vspace{0.5cm}
\textsuperscript{1} Wilhelm-Wundt-University\\\textsuperscript{2} Konstanz Business School}
\keywords{keywords\newline\indent Word count: X}
\usepackage{csquotes}
\usepackage{upgreek}
\captionsetup{font=singlespacing,justification=justified}

\usepackage{longtable}
\usepackage{lscape}
\usepackage{multirow}
\usepackage{tabularx}
\usepackage[flushleft]{threeparttable}
\usepackage{threeparttablex}

\newenvironment{lltable}{\begin{landscape}\begin{center}\begin{ThreePartTable}}{\end{ThreePartTable}\end{center}\end{landscape}}

\makeatletter
\newcommand\LastLTentrywidth{1em}
\newlength\longtablewidth
\setlength{\longtablewidth}{1in}
\newcommand{\getlongtablewidth}{\begingroup \ifcsname LT@\roman{LT@tables}\endcsname \global\longtablewidth=0pt \renewcommand{\LT@entry}[2]{\global\advance\longtablewidth by ##2\relax\gdef\LastLTentrywidth{##2}}\@nameuse{LT@\roman{LT@tables}} \fi \endgroup}


\DeclareDelayedFloatFlavor{ThreePartTable}{table}
\DeclareDelayedFloatFlavor{lltable}{table}
\DeclareDelayedFloatFlavor*{longtable}{table}
\makeatletter
\renewcommand{\efloat@iwrite}[1]{\immediate\expandafter\protected@write\csname efloat@post#1\endcsname{}}
\makeatother
\usepackage{lineno}

\linenumbers

\authornote{

Correspondence concerning this article should be addressed to First Author, Postal address. E-mail: \href{mailto:my@email.com}{\nolinkurl{my@email.com}}}

\abstract{

}

\begin{document}
\maketitle

\begin{verbatim}
## Warning: select_() is deprecated. 
## Please use select() instead
## 
## The 'programming' vignette or the tidyeval book can help you
## to program with select() : https://tidyeval.tidyverse.org
## This warning is displayed once per session.
\end{verbatim}

\begin{verbatim}
## Warning: group_by_() is deprecated. 
## Please use group_by() instead
## 
## The 'programming' vignette or the tidyeval book can help you
## to program with group_by() : https://tidyeval.tidyverse.org
## This warning is displayed once per session.
\end{verbatim}

\begin{verbatim}
## Warning: count_() is deprecated. 
## Please use count() instead
## 
## The 'programming' vignette or the tidyeval book can help you
## to program with count() : https://tidyeval.tidyverse.org
## This warning is displayed once per session.
\end{verbatim}

\begin{verbatim}
## Warning: Column `x`/`rn` has different attributes on LHS and RHS of join
\end{verbatim}

\begin{verbatim}
## Warning in names(x_bins)[1] <- c(x, "n", "variable_label"): number of items
## to replace is not a multiple of replacement length
\end{verbatim}

The current sample includes 178 participants and 89, including 30 female same-sex couples, 24 male same-sex couples, and 35 heterosexual couples. The sample was determined by excluding any participants who did not have data for the primary variables of interest: Parental Values and Social Competence. Thus, any dyad that does not have information from both partners in the relevant variables was excluded.

Out of the 178 participants, 95 are female and 83 are male. In this subset of participants,
42\% (n = 74) identify as \enquote{Exclusively gay/lesbian/homosexual},
11\% (n = 19) identify as \enquote{Mostly gay/lesbian/homosexual},
6\% (n = 10) identify as \enquote{Bisexual},
2\% (n = 3) identify as \enquote{Mostly heterosexual},
37\% (n = 66) identify as \enquote{Exclusively heterosexual},
2\% (n = 3) identify as \enquote{Queer}, and
1\% (n = 2) identify as \enquote{Pansexual}.

\begin{verbatim}
## # A tibble: 2 x 2
##   gender     n
##    <dbl> <int>
## 1      1    95
## 2      2    83
\end{verbatim}

\newpage

\hypertarget{references}{%
\section{References}\label{references}}

\begingroup
\setlength{\parindent}{-0.5in}
\setlength{\leftskip}{0.5in}

\hypertarget{refs}{}

\endgroup


\end{document}
